\include{settings}

\begin{document}

\include{titlepage}

\tableofcontents
\newpage

\section{Цель работы}

Познакомиться с основами проектирования схемы БД, языком описания сущностей и ограничений БД SQL-DDL.

\section{Программа работы}

\begin{enumerate}
	\item Самостоятельное изучение SQL-DDL.
	\item Создание скрипта БД в соответствии с согласованной схемой. Должны присутствовать первичные и внешние ключи, ограничения на диапазоны значений. Демонстрация скрипта преподавателю. 
	\item Создание скрипта, заполняющего все таблицы БД данными.
	\item Выполнение SQL-запросов, изменяющих схему созданной БД по заданию преподавателя. Демонстрация их работы преподавателю.
\end{enumerate}

\section{Теоретическая информация}

\textbf{Язык SQL} (Structured Query Language) -- язык структурированных запросов. Он позволяет формировать весьма сложные запросы к базам данных. В SQL определены два подмножества языка:
\begin{itemize}
	\item \textbf{SQL-DDL} (Data Definition Language) -- язык определения структур и ограничений целостности баз данных. Сюда относятся команды создания и удаления баз данных; создания, изменения и удаления таблиц; управления пользователями и т.д.
	\item \textbf{SQL-DML} (Data Manipulation Language) -- язык манипулирования данными: добавление, изменение, удаление и извлечение данных, управления транзакциями. Функции SQL-DML определяются первым словом в предложении (часто называемом запросом), которое является глаголом: \code{SELECT} (<<выбрать>>), \code{INSERT} (<<вставить>>), \code{UPDATE} (<<обновить>>), и \code{DELETE} (<<удалить>>). 
\end{itemize}

\section{Выполнение работы}

\subsection{Структура базы данных}

\begin{figure}[H]
	\centering
	\includegraphics[width=\linewidth]{../../lab1/pics/scheme}
	\caption{Структура базы данных}
\end{figure}

\subsection{Скрипт создания структуры базы данных}

\lstinputlisting{ianfix.sql}

\subsection{Скрипт заполнения таблиц тестовыми данными}

\lstinputlisting{insert.sql}

\subsection{ИНдивидуальное задание}
По заданию от преподавателя база данных была дополнена двумя таблицами preinstalled_software - для учёта необходимо ПО, и tournament - для учета турниров.

\begin{figure}[H]
	\centering
	\includegraphics[width=\linewidth]{../newdatabase}
	\caption{Измененная структура базы данных}
\end{figure}

\section{Выводы}

В ходе выполнения лабораторной работы были изучены основы создания скриптов на языке SQL. С помощью SQL-DDL были описаны структуры хранимой в баз данных информации. С использованием SQL-DML созданные структуры были заполнены конкретными данными. 

\end{document}
