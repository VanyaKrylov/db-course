\documentclass[a4paper,14pt]{extarticle}
\usepackage[utf8]{inputenc}
\usepackage[russian]{babel}
\usepackage{graphicx}
\usepackage{indentfirst}
\usepackage[top=0.8in, bottom=0.8in, left=0.8in, right=0.8in]{geometry}
\usepackage{pgfplots}
\usepackage{amsmath}
\usepackage{setspace}
\usepackage{titlesec}
\usepackage{pdfpages}
\usepackage{subcaption}
\usepackage{float}
\usepackage{longtable}
\usepackage{chngcntr}
\usepackage{pgfplots}
\usepackage{amsfonts}
\usepackage{hhline}
\usepackage{pgfplotstable}
\usepackage{multirow}
\usepackage{tikz,xcolor}
\usepackage{listingsutf8}
\usepackage{textcomp}
\usepackage{hyperref}


\lstset{
	language={SQL},
	backgroundcolor=\color{white},
	commentstyle=\color{blue},
	keywordstyle=\color{blue},
	numberstyle=\scriptsize\color{gray},
	stringstyle=\color{purple},
	basicstyle=\ttfamily\small,
	breakatwhitespace=false,
	breaklines=true,
	captionpos=b,
	keepspaces=true,
	numbers=left,
	numbersep=5pt,
	showspaces=false,
	showstringspaces=false,
	showtabs=false,
	tabsize=4,
	texcl=true,
	extendedchars=false,
	frame=single,
	morekeywords={IF, BIGSERIAL, SERIAL, TEXT, BIGINT, MONEY, BOOLEAN, REFERENCES},
	inputencoding=utf8/cp1251
}

\titleformat{\section}[hang]
  {\bfseries}
  {}
  {0em}
  {\hspace{-0.4pt}\large \thesection\hspace{0.6em}}
  
  
\titleformat{\subsection}[hang]
  {\bfseries}
  {}
  {0em}
  {\hspace{-0.4pt}\large \thesubsection\hspace{0.6em}}
  
\titleformat{\subsubsection}[hang]
  {\bfseries}
  {}
  {0em}
  {\hspace{-0.4pt}\large \thesubsubsection\hspace{0.6em}}

%\linespread{1.3} % полуторный интервал
%\renewcommand{\rmdefault}{ftm} % Times New Roman

\counterwithin{figure}{section}
\counterwithin{equation}{section}
\counterwithin{table}{section}

\begin{document}
\begin{titlepage}
\centering
Санкт-Петербургский политехнический университет Петра Великого \\
Институт компьютерных наук и технологий \\
Кафедра компьютерных систем и программных технологий \\
\vspace{5.5cm}

{\centering \textbf{Отчёт по лабораторной работе} \\ 
\vspace{0.15cm}
\textbf{Дисциплина}: Базы данных \\
\vspace{0.15cm}
\textbf{Тема}: Триггеры, вызовы процедур} \\

\vspace{5.5cm}

\begin{table}[H]
\begin{tabular}{p{\textwidth}@{}r}
{Выполнили студенты гр. 43501/3} \hfill 
\vspace{0.2cm}
Крылов И.С. \\ \hfill
\vspace{0.2cm}

Преподаватель \hfill 
\vspace{0.2cm}
Мяснов А.В \\ \hfill 
\vspace{0.2cm}

{} \hfill { <<\underline{\hspace{0.08\textwidth}}>> \underline{\hspace{0.2\textwidth}}2019 г.} \\
\end{tabular}
\end{table}
\vfill
{\centering Санкт-Петербург \\ 
\vspace{0.15cm}
2019}
\end{titlepage}

\section{Цель работы.}
Познакомить студентов с возможностями реализации более сложной обработки
данных на стороне сервера с помощью хранимых процедур и триггеров.

\section{Программа работы}
\begin{enumerate}
	\item Создание двух триггеров: один триггер для автоматического заполнения ключевого поля, второй триггер для контроля целостности данных в подчиненной таблице при удалении/изменении записей в главной таблице.
	\item Создание триггера в соответствии с индивидуальным заданием, полученным у преподавателя.
	\item Создание триггера в соответствии с индивидуальным заданием, вызывающего хранимую процедуру.
	\item Выкладывание скрипта с созданными сущностями в GitLab.
	\item Демонстрация результатов преподавателю.
\end{enumerate}

\section{Ход работы}
В соответствии с индивидуальным заданием было создано два триггера.

База была заполненна тестовыми данными с помощью генератора.

\subsection{Индивидальное задание}
Триггеры:
\begin{enumerate}
	\item При создании новой игры проверять нет ли игры у того же разрабочика с названием, отличающимся по дистанции Левенштейна более, чем на 3. Если есть - выбрасывать исключение и не давать добавлять.
	\item При формировании аппаратных требований к игре не должно быть ситуации, чтобы минимальные требования превышали оптимальные. Если производися попытка добавить или изменить требования так, что может привести к невыполнению требования, то выбрасывать исключение.
\end{enumerate}

\subsection{Добавление новой игры}
Был разработан sql-скрипт, создающий триггер, который активируется на добавление нового элемента в таблицу game и проверяет наличие игры того же разработчика с названием, отличающимся по дистанции Левенштейна менее чем на 3. Если такая игра присутсвует - выбрасывается исключение о невозможности добавить запись в таблицу.

\lstinputlisting{sql/LevenCheck.sql}

Созданный триггер был проверен на нескольких запросах. Можно утверждать, что триггер работает правильно.

\subsection{Добавление новых системных требований}
Проверяем наличие ситуации когда минимальные требования превышают оптимальные. При обнаружении такой ситуации выбрасываем исключение.

\lstinputlisting{sql/SysReqCheck.sql}

\section{Вывод}
В ходе лабораторной работы были улучшены навыки работы с sql и получены навыки разработки триггеров. Решения типа хранимых процедур и триггеров, позволяют перенести часть вычислений на сторону sql сервера. Учитывая, что он с подобными задачами справляется быстрее и при этом затрачивая меньшие ресурсы подобная практика считается хорошей. Главным недостатоком использования триггеров является то, что про них можно забыть и тогда разобраться, что происходит в базе данных может оказаться трудным.

\end{document}
