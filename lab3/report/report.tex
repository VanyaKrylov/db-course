\documentclass[a4paper,14pt]{extarticle}
\usepackage[utf8]{inputenc}
\usepackage[russian]{babel}
\usepackage{graphicx}
\usepackage{indentfirst}
\usepackage[top=0.8in, bottom=0.8in, left=0.8in, right=0.8in]{geometry}
\usepackage{pgfplots}
\usepackage{amsmath}
\usepackage{setspace}
\usepackage{titlesec}
\usepackage{pdfpages}
\usepackage{subcaption}
\usepackage{float}
\usepackage{longtable}
\usepackage{chngcntr}
\usepackage{pgfplots}
\usepackage{amsfonts}
\usepackage{hhline}
\usepackage{pgfplotstable}
\usepackage{multirow}
\usepackage{tikz,xcolor}
\usepackage{listingsutf8}
\usepackage{textcomp}
\usepackage{hyperref}


\lstset{
	language={SQL},
	backgroundcolor=\color{white},
	commentstyle=\color{blue},
	keywordstyle=\color{blue},
	numberstyle=\scriptsize\color{gray},
	stringstyle=\color{purple},
	basicstyle=\ttfamily\small,
	breakatwhitespace=false,
	breaklines=true,
	captionpos=b,
	keepspaces=true,
	numbers=left,
	numbersep=5pt,
	showspaces=false,
	showstringspaces=false,
	showtabs=false,
	tabsize=4,
	texcl=true,
	extendedchars=false,
	frame=single,
	morekeywords={IF, BIGSERIAL, SERIAL, TEXT, BIGINT, MONEY, BOOLEAN, REFERENCES},
	inputencoding=utf8/cp1251
}

\titleformat{\section}[hang]
  {\bfseries}
  {}
  {0em}
  {\hspace{-0.4pt}\large \thesection\hspace{0.6em}}
  
  
\titleformat{\subsection}[hang]
  {\bfseries}
  {}
  {0em}
  {\hspace{-0.4pt}\large \thesubsection\hspace{0.6em}}
  
\titleformat{\subsubsection}[hang]
  {\bfseries}
  {}
  {0em}
  {\hspace{-0.4pt}\large \thesubsubsection\hspace{0.6em}}

%\linespread{1.3} % полуторный интервал
%\renewcommand{\rmdefault}{ftm} % Times New Roman

\counterwithin{figure}{section}
\counterwithin{equation}{section}
\counterwithin{table}{section}

\begin{document}
\begin{titlepage}
\centering
Санкт-Петербургский политехнический университет Петра Великого \\
Институт компьютерных наук и технологий \\
Кафедра компьютерных систем и программных технологий \\
\vspace{5.5cm}

{\centering \textbf{Отчёт по лабораторной работе} \\ 
\vspace{0.15cm}
\textbf{Дисциплина}: Базы данных \\
\vspace{0.15cm}
\textbf{Тема}: Генератор тестовых данных} \\

\vspace{5.5cm}

\begin{table}[H]
\begin{tabular}{p{\textwidth}@{}r}
{Выполнил студент гр. 43501/3} \hfill 
\vspace{0.2cm}
Крылов И.С. \\ \hfill
\vspace{0.2cm}

Преподаватель \hfill 
\vspace{0.2cm}
Мяснов А.В \\ \hfill 
\vspace{0.2cm}

{} \hfill { <<\underline{\hspace{0.08\textwidth}}>> \underline{\hspace{0.2\textwidth}}2018 г.} \\
\end{tabular}
\end{table}
\vfill
{\centering Санкт-Петербург \\ 
\vspace{0.15cm}
2018}
\end{titlepage}

\section{Цель работы.}
Сформировать набор данных, позволяющий производить операции на реальных объемах данных.

\section{Программа работы}
\begin{enumerate}
\item Реализация в виде программы параметризуемого генератора, который позволит сформировать набор связанных данных в каждой таблице.
\item Частные требования к генератору, набору данных и результирующему набору данных:
\begin{itemize}
\item количество записей в справочных таблицах должно соответствовать ограничениям предметной области
\item количество записей в таблицах, хранящих информацию об объектах или субъектах должно быть параметром генерации
\item значения для внешних ключей необходимо брать из связанных таблиц
\end{itemize}
\end{enumerate}


\section{Ход работы}
Генератор данных для базы данных был реализован на языке Java версии 8. Выбор обусловлен наличием обширного выбора библиотек, в том числе библиотека поддерживающая работу с базой данных PostgreSQL. Задание было выполнено в среде разработки Intellij IDEA.

\subsection{Реализация генератора}
СТруктурно, проект был разделён на четыре класса, которые наиболее улобный образом реализуют работу генератора.
\begin{itemize}
\item Класс par задаёт параметры работы генератора
\item Класс Generator реализует генерацию случайных значений в требуемой формате
\item Класс func реализует основную функциональность генератора, запоняя строки таблиц базы данных случайно сгенерированными значениями.
\item Класс Service используется для запуска приложения.
\end{itemize}

Код генератора приведен \href{http://gitlab.icc.spbstu.ru/IvanKrylov/video-games-store/tree/master}{GitLab}.


\section{Вывод}
В резльтате данной работы был разработан генератор тестовых данных на яызке Java. Было осущественно подключение к базе данных PostgreSQL и последующее заполнение записей таблиц базы данных. ВНешние ключи связаны с первичными из связанных таблиц. Количество данных, а так же сама база данных задаются как параметры работы генератора.

\end{document}
