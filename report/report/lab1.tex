\include{settings}

\begin{document}

\include{titlepage}

\tableofcontents
\newpage

\section{Цель работы}

Познакомиться с основами проектирования схемы БД, способами организации данных в SQL-БД.

\section{Программа работы}

\begin{enumerate}
	\item Создание проекта для работы в GitLab.
	\item Выбор задания (предметной области), описание набора данных и требований к хранимым данным в свободном формате в Wiki своего проекта в GitLab.
	\item Формирование в свободном формате (предпочтительно в виде графической схемы) схемы БД, соответствующей заданию. Должно получиться не менее 7 таблиц.
	\item Согласование с преподавателем схемы БД. Обоснование принятых решений и соответствия требованиям выбранного задания. 
	\item Выкладывание схемы БД в свой проект в GitLab.
	\item Демонстрация результатов преподавателю.
\end{enumerate}

\section{Выполнение работы}


\subsection{Выбор предметной области}

Для выполнения работы была выбрана тема задания онлайн магазина видеоигр. База данных хранит общие сведения об игре, предоставляющие краткое описание самой игры, вместе с системными требованиями к оборудованию на котором возможен запуск игры.

\subsection{Описание таблиц}

В процессе проектирования схемы базы данных были выделены следующие сущности:
\begin{itemize}
	\item \code{game} -- хранит общую информацию об игреЮ такую как: \code{price} - стоимость игры, системные требования \code{system_requirements}, \code{developer} - разработчика игры, \code{release_date} - дата выпуска.
	
	\item \code{system_requirements} -- хранит информацию о минимальных \code{minimal_requirements} и оптимальных \code{optimal_requirements} системных требованиях.
\end{itemize}
	

\subsection{Структура базы данных}

\begin{figure}[H]
	\centering
	\includegraphics[width=\linewidth]{../pics/scheme.png}
\end{figure}

\section{Выводы}

В ходе выполнения данной работы была спроектирована и согласована с преподавателем база данных для онлайн магазина видеоигр. Было выделено несколько основные сущностей, таких как игра, разработчик и системные требования, а также множество вспомогательных таблиц для хранения информации о дополнительном игровом контенте и подробных системных требований.

\end{document}