\documentclass[a4paper,14pt]{extarticle}
\usepackage[utf8]{inputenc}
\usepackage[russian]{babel}
\usepackage{graphicx}
\usepackage{indentfirst}
\usepackage[top=0.8in, bottom=0.8in, left=0.8in, right=0.8in]{geometry}
\usepackage{pgfplots}
\usepackage{amsmath}
\usepackage{setspace}
\usepackage{titlesec}
\usepackage{pdfpages}
\usepackage{subcaption}
\usepackage{float}
\usepackage{longtable}
\usepackage{chngcntr}
\usepackage{pgfplots}
\usepackage{amsfonts}
\usepackage{hhline}
\usepackage{pgfplotstable}
\usepackage{multirow}
\usepackage{tikz,xcolor}
\usepackage{listingsutf8}
\usepackage{textcomp}
\usepackage{hyperref}


\lstset{
	language={SQL},
	backgroundcolor=\color{white},
	commentstyle=\color{blue},
	keywordstyle=\color{blue},
	numberstyle=\scriptsize\color{gray},
	stringstyle=\color{purple},
	basicstyle=\ttfamily\small,
	breakatwhitespace=false,
	breaklines=true,
	captionpos=b,
	keepspaces=true,
	numbers=left,
	numbersep=5pt,
	showspaces=false,
	showstringspaces=false,
	showtabs=false,
	tabsize=4,
	texcl=true,
	extendedchars=false,
	frame=single,
	morekeywords={IF, BIGSERIAL, SERIAL, TEXT, BIGINT, MONEY, BOOLEAN, REFERENCES},
	inputencoding=utf8/cp1251
}

\titleformat{\section}[hang]
  {\bfseries}
  {}
  {0em}
  {\hspace{-0.4pt}\large \thesection\hspace{0.6em}}
  
  
\titleformat{\subsection}[hang]
  {\bfseries}
  {}
  {0em}
  {\hspace{-0.4pt}\large \thesubsection\hspace{0.6em}}
  
\titleformat{\subsubsection}[hang]
  {\bfseries}
  {}
  {0em}
  {\hspace{-0.4pt}\large \thesubsubsection\hspace{0.6em}}

%\linespread{1.3} % полуторный интервал
%\renewcommand{\rmdefault}{ftm} % Times New Roman

\counterwithin{figure}{section}
\counterwithin{equation}{section}
\counterwithin{table}{section}

\begin{document}
\begin{titlepage}
\centering
Санкт-Петербургский политехнический университет Петра Великого \\
Институт компьютерных наук и технологий \\
Кафедра компьютерных систем и программных технологий \\
\vspace{5.5cm}

{\centering \textbf{Отчёт по лабораторной работе} \\ 
\vspace{0.15cm}
\textbf{Дисциплина}: Базы данных \\
\vspace{0.15cm}
\textbf{Тема}: Язык SQL-DML} \\

\vspace{5.5cm}

\begin{table}[H]
\begin{tabular}{p{\textwidth}@{}r}
{Выполнил студент гр. 43501/3} \hfill 
\vspace{0.2cm}
Крылов И.С. \\ \hfill
\vspace{0.2cm}

Преподаватель \hfill 
\vspace{0.2cm}
Мяснов А.В \\ \hfill 
\vspace{0.2cm}

{} \hfill { <<\underline{\hspace{0.08\textwidth}}>> \underline{\hspace{0.2\textwidth}}2018 г.} \\
\end{tabular}
\end{table}
\vfill
{\centering Санкт-Петербург \\ 
\vspace{0.15cm}
2018}
\end{titlepage}

\section{Цель работы.}
Познакомитьсяс языком создания запросов управления данными SQL-DML.

\section{Программа работы}
\begin{enumerate}
\item Изучение SQL-DML.
\item Выполнение всех запросов из списка стандартных запросов. Демонстрация результатов преподавателю.
\item Получение у преподавателя и реализация SQL-запросов в соответствии с индивидуальным заданием. Демонстрация результатов преподавателю.
\item Сохранение в БД выполненных запросов SELECT в виде представлений, запросов INSERT, UPDATE или DELETE -- в виде ХП. Выкладывание скрипта в GitLab.
\end{enumerate}

\section{Список стандартных запросов}
\begin{itemize}
\item Сделайте выборку всех данных из каждой таблицы
\item Сделайте выборку данных из одной таблицы при нескольких условиях, с использованием логических операций, LIKE, BETWEEN, IN (не менее 3-х разных примеров)
\item Создайте в запросе вычисляемое поле
\item Сделайте выборку всех данных с сортировкой по нескольким полям
\item Создайте запрос, вычисляющий несколько совокупных характеристик таблиц
\item Сделайте выборку данных из связанных таблиц (не менее двух примеров)
\item Создайте запрос, рассчитывающий совокупную характеристику с использованием группировки, наложите ограничение на результат группировки
\item Придумайте и реализуйте пример использования вложенного запроса
\item С помощью оператора INSERT добавьте в каждую таблицу по одной записи
\item С помощью оператора UPDATE измените значения нескольких полей у всех записей, отвечающих заданному условию
\item С помощью оператора DELETE удалите запись, имеющую максимальное (минимальное) значение некоторой совокупной характеристики
\item С помощью оператора DELETE удалите записи в главной таблице, на которые не ссылается подчиненная таблица (используя вложенный запрос)
\end{itemize}

\section{Ход работы}

База была заполненна тестовыми данными с помощью генератора.

\subsection{Выполнение стандартных запросов}

Для того, чтобы сделать выборку всех данных из каждой таблицы воспользуемся скриптом:
\lstinputlisting{../sql/db1.sql}

Получим данные из таблиц $developer$ и $tournament$ при нескольких условиях, с использованием логических операций, LIKE, BETWEEN, IN:
\lstinputlisting{../sql/db2.sql}

Используем в запросе вычисляемое поле, чтобы определить сколько турниров с призом = 4:
\lstinputlisting{../sql/db3.sql}

Организуем выборку данных с сотрировкой по нескольким полям.
Сортировать данные будем из таблицы $tournament$ по полям $tournament\_name$ и $tournament\_xprize$:
\lstinputlisting{../sql/db4.sql}
Таким образом получим турниры по убыванию призового фонда.

Для вычисления совокупных характеристик таблицы разработаем sql-скрипт:
\lstinputlisting{../sql/db5.sql}

Сделаем выборку данных из связаных таблиц.
\lstinputlisting{../sql/db6.sql}

Выведем память gpu по возрастанию частоты значения в БД:
\lstinputlisting{../sql/db7.sql}

Вложенный запрос, который выводит 64 разрядные системы и процессоры с ОП 2Гб :
\lstinputlisting{../sql/db8.sql}

Был разработан скрипт, который добавляет в таблицу по одной записи: 
\lstinputlisting{../sql/db9.sql}

Создан скрипт, который изменяет значение в базе данных:\\ 
\lstinputlisting{../sql/db10.sql}

Удалим игру с наибольшей стомостью:
\lstinputlisting{../sql/db11.sql}

Удалим системные требования, не привязанные к игре:
\lstinputlisting{../sql/db12.sql}

\section{Запросы по индивидуальному заданию}
\subsection{Задание}
Реализовать запросы к БД:

\begin{enumerate}
\item Вывести процессоры, производительности которых хватает для запуска более половины игр из всего перечня. Производительность процессора измерять как произведение тактовой частоты на количество ядер.
\item Вывести игры, для которых суммарная стоимость дополнительного контента более, чем в 3 раза превышает стоимость игры.
\end{enumerate}

\subsection{Процессоры}
Был разработан sql-скрипт, который выводит процессоры с производительностью больше чем требуется для половины имеющихся игр.

\lstinputlisting{../sql/db14.sql}

Для каждой игры считается необходимая мощность процессора, после чего составляется упорядоченная по требуемой мощности для игры таблица, берётся медианное значение и сравнивается для каждого из процессоров.


\subsection{Доп контент}
Был разработан sql-скрипт.

\lstinputlisting{../sql/db13.sql}

Для каждой игры была посчитана общая сумма дополнительного контента, после чего это значение было сравнено с о стоимостью самой игры.

\section{Представления и хранимые процедуры}
Из запросов сделаных в предидущих пунктах сформируем представления:

\lstinputlisting{../sql/view.sql}

Оформить запросы на изменение базы данных, как хранимые процедуры не получится. Это связано с тем, что хранимые процедуры  достпуны только начиная с PostgreSQL11. Поэтому будем использовать функции:
\lstinputlisting{../sql/func.sql}

\section{Вывод}
В ходе лабораторной работы были улучшены навыки работы с sql-dml и улучшено общее понимание функционирования баз данных. К основным запросам sql-dml относятся SELECT, INSERT, UPDATE, DELETE. В данной работе наиболее подробно рассмотрена работа  с SELECT. В запросов этого типа были установлены различные ограничения, вложенные конструкции, объединения. По результатам использования sql-dml можно сказать, что это мощный инструмент, который позволяет легко и элегантно решить поставленную задачу.

\end{document}
